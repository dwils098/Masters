\documentclass{article}
\usepackage[latin1]{inputenc}

\usepackage{geometry}

\begin{document}

\title{An Architecture for a Fully Decentralized Peer-to-Peer Collaborative Computing
  Platform}
\author{Dany Wilson -- Dr. St\'ephane Som\'e}
\date

\maketitle

We present an architecture for a fully decentralized peer-to-peer collaborative computing
platform, offering services similar to Cloud Service Provider's Platform-as-a-Service
(PaaS) model, using volunteered resources rather than dedicated resources. This thesis is
motivated by three research questions: (1) Is it possible to build a peer-to-peer
collaborative system using a fully decentralized infrastructure relying only on
volunteered resources?, (2) How can light virtualization be used to mitigate the
complexity inherent to the volunteered resources?, and (3) What are the minimal
requirements for a computing platform similar to the PaaS cloud computing platform?
Previous research on the \emph{volunteer cloud computing} paradigm, focused on providing
various service models and even full-fledged volunteer cloud computing
infrastructures. Whereas previous literature on \emph{peer-to-peer collaborative systems}
expressed the requirements inherent to the peer-to-peer resource collaboration
problem. Bridging these two fields of research, we evaluate two major projects offering a
volunteer cloud computing infrastructure, \emph{Cloud@Home} and \emph{Peer-to-Peer Cloud
  System}, using the requirements identified for peer-to-peer collaborative systems. This
thesis shifts the perspective from peer-to-peer collaborative systems, to their use as the
underlying foundation of volunteer cloud computing infrastructures.

The architecture proposed is composed of three layers: the \emph{Network layer}, the
\emph{Virtual layer}, and the \emph{Application layer}. We propose to implement the
\emph{Network layer} using two novel abstractions: the \emph{Ring}, for the public
peer-to-peer networking primitive, and the \emph{Fellowship(s)}, for the private
application networking primitive. We also propose to use \emph{light virtualization}
technologies, or containers, to provide a uniform abstraction of the contributing
resources and to isolate the host environment from the contributed environment. Then,
we propose a minimal API specification for this computing platform, which is also
applicable to PaaS computing platforms.

We showcase the architecture with a proof of concept, a distributed web calculator, and by
presenting a more complex application for \emph{Multi-Document Text Summarization using
  Genetic Algorithm}. 

The findings of this thesis corroborate the hypothesis that peer-to-peer collaborative
systems can be used as a basis for developing volunteer cloud computing
infrastructures. We outline the implications of using light virtualization as an integral
virtualization primitive in public distributed computing platform. Finally, this thesis
lays out a starting point for most volunteer cloud computing infrastructure development
effort, because it circumscribes the essential requirements and presents solutions to
mitigate the complexities inherent to this paradigm.

\end{document}