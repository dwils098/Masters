\documentclass{article}
\usepackage[latin1]{inputenc}

\begin{document}
\title{An Architecture for a Fully De-Centralized Peer-to-Peer Volunteer Cloud Computing Infrastructure}
\author{Dany Wilson -- Dr. St\'ephane Som\'e} 
\date{January 15 2015}

\maketitle 

Cloud Computing is becoming more popular everyday, and new application of this paradigm
are constantly being discovered. It provides a lot of advantages of an on-premise
solution, at a fraction of the cost. But much more than just financial benefits, there are
also a lot of different benefits, such as augmented availability, geo-distribution of the
assets, augmented reliability, and limitless scalability (theoretically). A new paradigm
is emerging from this technology, where it attempts to combine the collaborative nature of
Volunteer Computing projects such as SETI@Home and the Cloud Computing infrastructure to
build a collaborative Cloud Computing Infrastructure. In this paper we will present the
current state of the art of this new paradigm and propose a fully de-centralized
peer-to-peer volunteer cloud computing infrastructure. Also, we lay out the map to build a
full-fledged cloud computing infrastructure, but only provide a proof of concept
implementation for a Platform-as-a-Service infrastructure. A comprehensive study of all
the major problems relating to the construction of a collaborative system, and the
challenges present when attempting to provide an homogeneous perspective of a set of
heterogeneous resources is fundamental. Thus, we are trying to clarify the current
misconceptions with respect to the VCC paradigm and generate a tentative ontological
representation of it. Leveraging Linux capabilities in terms of virtualization and
containers (or operating-system level virtualization), we provide a comprehensive API that
encompass the major services necessary from the PaaS level. This API constitute a
"barebone", or minimalist, specification of the requirements from a PaaS level and is our
second major contribution to the paradigm. This technology aims at giving back the control
to the user and liberate them from the oligarchy that governs the cloud computing service
providers market. It also aims at recycling the idling resources around the World and to
make use of them, by making them available to the World-wide community. Finally, with the
current events with regards to whistleblowers and the different information leaking
scandals World-wide; there is a concern in terms of security since some corporation
considers the user's data as an asset and willingly disclosed them to any party as they
wish (under political pressure or not). There is an eminent need to understand that this
information belongs to the user, and thus responsibility should be reverted to the user.

\end{document}
